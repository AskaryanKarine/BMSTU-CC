%\pagenumbering{arabic}
\section*{ВВЕДЕНИЕ}
\phantomsection
\addcontentsline{toc}{section}{ВВЕДЕНИЕ}
 
Компилятор~---~это программная система, которая преобразует код, написанный на языке программирования, в форму, пригодную для выполнения на компьютере~\cite{aho2003}. 

Современный мир зависит от языков программирования, поскольку все программное обеспечение на компьютерах написано на том или ином языке, и компиляторы играют ключевую роль в этом процессе~\cite{aho2003}.

\textbf{Целью} данной работы является разработка компилятора для языка\\КуМир. Компилятор должен выполнять чтение текстового файла, содержащего код на языке КуМир и генерировать на выходе LLVM IR программы, пригодный для запуска.

Для достижения поставленной цели необходимо решить следующие \textbf{задачи}:
\begin{enumerate}
	\item проанализировать грамматику языка КуМир;
	\item изучить существующие средства для анализа исходного кода программы, системы генерации низкоуровневого кода;
	\item реализовать прототип компиляторы;
	\item провести тестирование компилятора.
\end{enumerate}


\newpage