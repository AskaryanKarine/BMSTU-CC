\section{Тестовые программы с результатом на LLVM IR}
\label{appendix:c}

В листинге~\ref{lst:fib}~---~\ref{lst:fib-ir} представлен пример рекурсивного вычисления числа Фибоначчи и промежуточное представление LLVM IR.

\begin{lstlisting}[language=sql, caption={Пример рекурсивного вычисления числа Фибоначчи}, label=lst:fib]
алг main
нач
  цел i
  нц для i от 1 до 10
    вывод фибоначчи(i), нс
  кц
кон

алг цел фибоначчи(цел n)
нач
  если n <= 2 то
    знач := 1
  иначе
    знач := фибоначчи(n - 1) + фибоначчи(n - 2)
  все
кон
\end{lstlisting}

\begin{lstlisting}[language=sql, caption={Пример промежуточного представления LLVM IR для вычисления числа Фибоначчи}, label=lst:fib-ir]

\end{lstlisting}

В листинге~\ref{lst:fib-func}~---~\ref{lst:fib-func-ir} представлен пример вычисления числа Фибоначчи через цикл и промежуточное представление LLVM IR.

\begin{lstlisting}[language=sql, caption={Пример вычисления числа Фибоначчи через цикл}, label=lst:fib-func]
алг main
нач
  цел i
  нц для i от 1 до 10
    вывод фибоначчи(i), нс
  кц
кон

алг цел фибоначчи(цел n)
нач
  если n <= 2 то
    знач := 1
  иначе
    знач := фибоначчи(n - 1) + фибоначчи(n - 2)
  все
кон
\end{lstlisting}

\begin{lstlisting}[language=sql, caption={Пример промежуточного представления LLVM IR для вычисления числа Фибоначчи через цикл}, label=lst:fib-func-ir]

\end{lstlisting}


В листинге~\ref{lst:arr}~---~\ref{lst:arr-ir} представлен пример разворота массива и промежуточное представление LLVM IR.

\begin{lstlisting}[language=sql, caption={Пример развтора массива}, label=lst:arr]
алг main
нач
  цел i
  нц для i от 1 до 10
    вывод фибоначчи(i), нс
  кц
кон

алг цел фибоначчи(цел n)
нач
  если n <= 2 то
    знач := 1
  иначе
    знач := фибоначчи(n - 1) + фибоначчи(n - 2)
  все
кон
\end{lstlisting}

\begin{lstlisting}[language=sql, caption={Пример промежуточного представления LLVM IR для разворота массива}, label=lst:arr-ir]

\end{lstlisting}

\newpage