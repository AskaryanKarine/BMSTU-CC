\section{Тестовые программы с результатом на LLVM IR}
\label{appendix:c}

В листинге~\ref{lst:fib}~---~\ref{lst:fib-ir} представлен пример рекурсивного вычисления числа Фибоначчи и промежуточное представление LLVM IR.

\begin{lstlisting}[language=sql, caption={Пример рекурсивного вычисления числа Фибоначчи}, label=lst:fib]
алг main
нач
    цел a
    a := фибоначчи(5)
    вывод a, нс
кон

алг цел фибоначчи(цел n)
нач
    если n <= 2 то
        знач := 1
    иначе
        знач := фибоначчи(n - 1) + фибоначчи(n - 2)
    все
кон
\end{lstlisting}

\begin{lstlisting}[language=sql, caption={Пример промежуточного представления LLVM IR для рекурсивного вычисления числа Фибоначчи}, label=lst:fib-ir]
target triple = "arm64-apple-macosx14.0.0"

@.str.literal.0 = constant [2 x i8] c"\0A\00"
@.fmt.str.1 = constant [6 x i8] c"%d %s\00"

define i32 @main() {
entry:
	%0 = alloca i32
	%1 = alloca i32
	%2 = call i32 @fibonachchi_rus(i32 5)
	store i32 %2, i32* %1
	%3 = load i32, i32* %1
	%4 = getelementptr [2 x i8], [2 x i8]* @.str.literal.0, i32 0, i32 0
	%5 = getelementptr [6 x i8], [6 x i8]* @.fmt.str.1, i32 0, i32 0
	%6 = call i32 (i8*, ...) @printf(i8* %5, i32 %3, i8* %4)
	%7 = load i32, i32* %0
	ret i32 %7
}
\end{lstlisting}

\newpage
\setcounter{lstlisting}{1}
\begin{lstlisting}[language=go, firstnumber=last, caption={Пример промежуточного представления LLVM IR для рекурсивного вычисления числа Фибоначчи}]
define i32 @fibonachchi_rus(i32 %n) {
entry:
	%0 = alloca i32
	%1 = alloca i32
	store i32 %n, i32* %1
	%2 = load i32, i32* %1
	%3 = icmp sle i32 %2, 2
	br i1 %3, label %if.then.1, label %if.else.3

if.then.1:
	store i32 1, i32* %0
	br label %if.end.2

if.end.2:
	%4 = load i32, i32* %0
	ret i32 %4

if.else.3:
	%5 = load i32, i32* %1
	%6 = sub i32 %5, 1
	%7 = call i32 @fibonachchi_rus(i32 %6)
	%8 = load i32, i32* %1
	%9 = sub i32 %8, 2
	%10 = call i32 @fibonachchi_rus(i32 %9)
	%11 = add i32 %7, %10
	store i32 %11, i32* %0
	br label %if.end.2
}

declare i32 @printf(i8* %format, ...)
\end{lstlisting}

В листинге~\ref{lst:fib-func}~---~\ref{lst:fib-func-ir} представлен пример вычисления числа Фибоначчи через цикл и промежуточное представление LLVM IR.

\begin{lstlisting}[language=sql, caption={Пример вычисления числа Фибоначчи через цикл}, label=lst:fib-func]
алг FibIter
нач
    цел n
    цел a, b, i
    n := 5
    вывод 'n =', n, нс
    a := 0
    b := 1
\end{lstlisting}

\newpage
\setcounter{lstlisting}{2}
\begin{lstlisting}[language=go, firstnumber=last, caption={Пример вычисления числа Фибоначчи через цикл}]
    если n = 0 то
        вывод 'F(', n, ') =', a, нс
        выход
    иначе
        если n = 1 то
            вывод 'F(', n, ') =', b, нс
            выход
        все
    все
    нц для i от 2 до n
        b := a + b
        a := b - a
    кц
    вывод 'F(', n, ') =', b, нс
кон
\end{lstlisting}

\begin{lstlisting}[language=sql, caption={Пример промежуточного представления LLVM IR для вычисления числа Фибоначчи через цикл}, label=lst:fib-func-ir]
target triple = "arm64-apple-macosx14.0.0"

@.str.literal.0 = constant [4 x i8] c"n =\00"
@.str.literal.1 = constant [2 x i8] c"\0A\00"
@.fmt.str.2 = constant [9 x i8] c"%s %d %s\00"
@.str.literal.3 = constant [3 x i8] c"F(\00"
@.str.literal.4 = constant [4 x i8] c") =\00"
@.str.literal.5 = constant [2 x i8] c"\0A\00"
@.fmt.str.6 = constant [15 x i8] c"%s %d %s %d %s\00"
@.str.literal.7 = constant [3 x i8] c"F(\00"
@.str.literal.8 = constant [4 x i8] c") =\00"
@.str.literal.9 = constant [2 x i8] c"\0A\00"
@.fmt.str.10 = constant [15 x i8] c"%s %d %s %d %s\00"
@.str.literal.11 = constant [3 x i8] c"F(\00"
@.str.literal.12 = constant [4 x i8] c") =\00"
@.str.literal.13 = constant [2 x i8] c"\0A\00"
@.fmt.str.14 = constant [15 x i8] c"%s %d %s %d %s\00"

define i32 @main() {
entry:
	%0 = alloca i32
	%1 = alloca i32
	%2 = alloca i32
	%3 = alloca i32
	%4 = alloca i32
	store i32 5, i32* %1
	%5 = getelementptr [4 x i8], [4 x i8]* @.str.literal.0, i32 0, i32 0
	%6 = load i32, i32* %1
\end{lstlisting}

\newpage
\setcounter{lstlisting}{2}
\begin{lstlisting}[language=go, firstnumber=last, caption={Пример промежуточного представления LLVM IR для вычисления числа Фибоначчи через цикл}]
	%7 = getelementptr [2 x i8], [2 x i8]* @.str.literal.1, i32 0, i32 0
	%8 = getelementptr [9 x i8], [9 x i8]* @.fmt.str.2, i32 0, i32 0
	%9 = call i32 (i8*, ...) @printf(i8* %8, i8* %5, i32 %6, i8* %7)
	store i32 0, i32* %2
	store i32 1, i32* %3
	%10 = load i32, i32* %1
	%11 = icmp eq i32 %10, 0
	br i1 %11, label %if.then.1, label %if.else.3

if.then.1:
	%12 = getelementptr [3 x i8], [3 x i8]* @.str.literal.3, i32 0, i32 0
	%13 = load i32, i32* %1
	%14 = getelementptr [4 x i8], [4 x i8]* @.str.literal.4, i32 0, i32 0
	%15 = load i32, i32* %2
	%16 = getelementptr [2 x i8], [2 x i8]* @.str.literal.5, i32 0, i32 0
	%17 = getelementptr [15 x i8], [15 x i8]* @.fmt.str.6, i32 0, i32 0
	%18 = call i32 (i8*, ...) @printf(i8* %17, i8* %12, i32 %13, i8* %14, i32 %15, i8* %16)
	ret i32 0

if.end.2:
	%19 = load i32, i32* %1
	%20 = alloca i32
	store i32 2, i32* %20
	br label %loop.cond.6

if.else.3:
	%21 = load i32, i32* %1
	%22 = icmp eq i32 %21, 1
	br i1 %22, label %if.then.4, label %if.end.5

if.then.4:
	%23 = getelementptr [3 x i8], [3 x i8]* @.str.literal.7, i32 0, i32 0
	%24 = load i32, i32* %1
	%25 = getelementptr [4 x i8], [4 x i8]* @.str.literal.8, i32 0, i32 0
	%26 = load i32, i32* %3
	%27 = getelementptr [2 x i8], [2 x i8]* @.str.literal.9, i32 0, i32 0
	%28 = getelementptr [15 x i8], [15 x i8]* @.fmt.str.10, i32 0, i32 0
	%29 = call i32 (i8*, ...) @printf(i8* %28, i8* %23, i32 %24, i8* %25, i32 %26, i8* %27)
	ret i32 0

if.end.5:
	br label %if.end.2

loop.cond.6:
	%30 = load i32, i32* %20
	%31 = icmp sle i32 %30, %19
	br i1 %31, label %loop.body.7, label %loop.end.9

loop.body.7:
	%32 = load i32, i32* %2
	%33 = load i32, i32* %3
	%34 = add i32 %32, %33
	store i32 %34, i32* %3
	%35 = load i32, i32* %3
	%36 = load i32, i32* %2
	%37 = sub i32 %35, %36
	store i32 %37, i32* %2
	br label %loop.step.8

loop.step.8:
	%38 = load i32, i32* %20
	%39 = add i32 %38, 1
	store i32 %39, i32* %20
	br label %loop.cond.6

loop.end.9:
	%40 = getelementptr [3 x i8], [3 x i8]* @.str.literal.11, i32 0, i32 0
	%41 = load i32, i32* %1
	%42 = getelementptr [4 x i8], [4 x i8]* @.str.literal.12, i32 0, i32 0
	%43 = load i32, i32* %3
	%44 = getelementptr [2 x i8], [2 x i8]* @.str.literal.13, i32 0, i32 0
	%45 = getelementptr [15 x i8], [15 x i8]* @.fmt.str.14, i32 0, i32 0
	%46 = call i32 (i8*, ...) @printf(i8* %45, i8* %40, i32 %41, i8* %42, i32 %43, i8* %44)
	%47 = load i32, i32* %0
	ret i32 %47
}

declare i32 @printf(i8* %format, ...)
\end{lstlisting}


В листинге~\ref{lst:arr}~---~\ref{lst:arr-ir} представлен пример разворота массива и промежуточное представление LLVM IR.

\begin{lstlisting}[language=sql, caption={Пример программы для развтора массива}, label=lst:arr]
алг Объявление массивов
нач
    целтаб A[1:5]
    цел N
    N := 5
    A[1] := 1
    A[2] := 2
    A[3] := 3
\end{lstlisting}

\newpage
\setcounter{lstlisting}{3}
\begin{lstlisting}[language=go, firstnumber=last, caption={Пример программы для развтора массива}]
    A[4] := 4
    A[5] := 5

    Реверс(A, N)
    вывод 'После реверса', нс
    ВыводМассива(A, N)
кон

алг ВыводМассива(целтаб A[1:5], цел длина)
нач
    нц для i от 1 до длина
        вывод A[i], нс
    кц
кон

алг Реверс(целтаб arr[1:5], цел N)
нач
    цел c
    нц для i от 1 до N div 2
        c:= arr[i]
        arr[i]:= arr[N+1-i]
        arr[N+1-i]:= c
    кц
кон
\end{lstlisting}

\begin{lstlisting}[language=sql, caption={Пример промежуточного представления LLVM IR для разворота массива}, label=lst:arr-ir]
target triple = "arm64-apple-macosx14.0.0"

@.str.literal.0 = constant [26 x i8] c"\D0\9F\D0\BE\D1\81\D0\BB\D0\B5 \D1\80\D0\B5\D0\B2\D0\B5\D1\80\D1\81\D0\B0\00"
@.str.literal.1 = constant [2 x i8] c"\0A\00"
@.fmt.str.2 = constant [6 x i8] c"%s %s\00"
@.str.literal.3 = constant [2 x i8] c"\0A\00"
@.fmt.str.4 = constant [6 x i8] c"%d %s\00"

define i32 @main() {
entry:
	%0 = alloca i32
	%1 = alloca [5 x i32]
	%2 = alloca i32
	store i32 5, i32* %2
	%3 = sub i32 1, 1
	%4 = getelementptr [5 x i32], [5 x i32]* %1, i32 0, i32 %3
	store i32 1, i32* %4
	%5 = sub i32 2, 1
\end{lstlisting}

\newpage
\setcounter{lstlisting}{4}
\begin{lstlisting}[language=go, firstnumber=last, caption={Пример промежуточного представления LLVM IR для разворота массива}]
	%6 = getelementptr [5 x i32], [5 x i32]* %1, i32 0, i32 %5
	store i32 2, i32* %6
	%7 = sub i32 3, 1
	%8 = getelementptr [5 x i32], [5 x i32]* %1, i32 0, i32 %7
	store i32 3, i32* %8
	%9 = sub i32 4, 1
	%10 = getelementptr [5 x i32], [5 x i32]* %1, i32 0, i32 %9
	store i32 4, i32* %10
	%11 = sub i32 5, 1
	%12 = getelementptr [5 x i32], [5 x i32]* %1, i32 0, i32 %11
	store i32 5, i32* %12
	%13 = load i32, i32* %2
	call void @Revers_rus([5 x i32]* %1, i32 %13)
	%14 = getelementptr [26 x i8], [26 x i8]* @.str.literal.0, i32 0, i32 0
	%15 = getelementptr [2 x i8], [2 x i8]* @.str.literal.1, i32 0, i32 0
	%16 = getelementptr [6 x i8], [6 x i8]* @.fmt.str.2, i32 0, i32 0
	%17 = call i32 (i8*, ...) @printf(i8* %16, i8* %14, i8* %15)
	%18 = load i32, i32* %2
	call void @VyvodMassiva_rus([5 x i32]* %1, i32 %18)
	%19 = load i32, i32* %0
	ret i32 %19
}

define void @VyvodMassiva_rus([5 x i32]* %A, i32 %dlina_rus) {
entry:
	%0 = alloca i32
	store i32 %dlina_rus, i32* %0
	%1 = load i32, i32* %0
	%2 = alloca i32
	store i32 1, i32* %2
	br label %loop.cond.1

loop.cond.1:
	%3 = load i32, i32* %2
	%4 = icmp sle i32 %3, %1
	br i1 %4, label %loop.body.2, label %loop.end.4

loop.body.2:
	%5 = load i32, i32* %2
	%6 = sub i32 %5, 1
	%7 = getelementptr [5 x i32], [5 x i32]* %A, i32 0, i32 %6
	%8 = load i32, i32* %7
	%9 = getelementptr [2 x i8], [2 x i8]* @.str.literal.3, i32 0, i32 0
	%10 = getelementptr [6 x i8], [6 x i8]* @.fmt.str.4, i32 0, i32 0
	%11 = call i32 (i8*, ...) @printf(i8* %10, i32 %8, i8* %9)
\end{lstlisting}

\newpage
\setcounter{lstlisting}{4}
\begin{lstlisting}[language=go, firstnumber=last, caption={Пример промежуточного представления LLVM IR для разворота массива}]
	br label %loop.step.3

  loop.step.3:
	%12 = load i32, i32* %2
	%13 = add i32 %12, 1
	store i32 %13, i32* %2
	br label %loop.cond.1

loop.end.4:
	ret void
}

define void @Revers_rus([5 x i32]* %arr, i32 %N) {
entry:
	%0 = alloca i32
	store i32 %N, i32* %0
	%1 = alloca i32
	%2 = load i32, i32* %0
	%3 = sdiv i32 %2, 2
	%4 = alloca i32
	store i32 1, i32* %4
	br label %loop.cond.1

loop.cond.1:
	%5 = load i32, i32* %4
	%6 = icmp sle i32 %5, %3
	br i1 %6, label %loop.body.2, label %loop.end.4

loop.body.2:
	%7 = load i32, i32* %4
	%8 = sub i32 %7, 1
	%9 = getelementptr [5 x i32], [5 x i32]* %arr, i32 0, i32 %8
	%10 = load i32, i32* %9
	store i32 %10, i32* %1
	%11 = load i32, i32* %4
	%12 = sub i32 %11, 1
	%13 = getelementptr [5 x i32], [5 x i32]* %arr, i32 0, i32 %12
	%14 = load i32, i32* %0
	%15 = add i32 %14, 1
	%16 = load i32, i32* %4
	%17 = sub i32 %15, %16
	%18 = sub i32 %17, 1
	%19 = getelementptr [5 x i32], [5 x i32]* %arr, i32 0, i32 %18
	%20 = load i32, i32* %19
	store i32 %20, i32* %13
\end{lstlisting}

\newpage
\setcounter{lstlisting}{4}
\begin{lstlisting}[language=go, firstnumber=last, caption={Пример промежуточного представления LLVM IR для разворота массива}]
	%21 = load i32, i32* %0
	%22 = add i32 %21, 1
	%23 = load i32, i32* %4
  %24 = sub i32 %22, %23
	%25 = sub i32 %24, 1
	%26 = getelementptr [5 x i32], [5 x i32]* %arr, i32 0, i32 %25
	%27 = load i32, i32* %1
	store i32 %27, i32* %26
	br label %loop.step.3

loop.step.3:
	%28 = load i32, i32* %4
	%29 = add i32 %28, 1
	store i32 %29, i32* %4
	br label %loop.cond.1

loop.end.4:
	ret void
}

declare i32 @printf(i8* %format, ...)
\end{lstlisting}

\newpage