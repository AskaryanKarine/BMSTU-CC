\renewcommand\appendixname{ПРИЛОЖЕНИЕ}

\titleformat{\section}[block]
{\bfseries\large}
{\appendixname~\thesection\centering\\}
{0pt}
{\centering}

\renewcommand{\thesection}{\Asbuk{section}}

\section{Грамматика языка КуМир}
\label{appendix:a}

В листинге~\ref{lst:kafka-click} представлен лексер грамматики в формате ANTRL.

\begin{lstlisting}[language=sql, caption={Лексер грамматики в формате ANTRL}, label=lst:kafka-click]
lexer grammar KumirLexer;

options { caseInsensitive = true; }

ALG_HEADER          : 'алг';
ALG_BEGIN           : 'нач';
ALG_END             : 'кон';
LOOP                : 'нц';
ENDLOOP_COND        : ('кц' WS 'при' | 'кц_при');
ENDLOOP             : 'кц';
IF                  : 'если';
THEN                : 'то';
ELSE                : 'иначе';
FI                  : 'все';
SWITCH              : 'выбор';
CASE                : 'при';
OUTPUT              : 'вывод';
ASSIGN              : ':=';
EXIT                : 'выход';
FOR                 : 'для';
WHILE               : 'пока';
TIMES               : 'раз';
FROM                : 'от';
TO                  : 'до';
STEP                : 'шаг';
NEWLINE_CONST       : 'нс';
NOT                 : 'не';
AND                 : 'и';
OR                  : 'или';
OUT_PARAM           : 'рез';
IN_PARAM            : 'арг';
INOUT_PARAM         : ('аргрез' | 'арг' WS 'рез' | 'арг_рез');
RETURN_VALUE        : 'знач';

INTEGER_TYPE        : 'цел';
REAL_TYPE           : 'вещ';
BOOLEAN_TYPE        : 'лог';
CHAR_TYPE           : 'сим';
STRING_TYPE         : 'лит';
TABLE_SUFFIX        : 'таб';
INTEGER_ARRAY_TYPE  : ('цел' WS? 'таб' | 'цел_таб');
REAL_ARRAY_TYPE     : ('вещ' WS? 'таб' | 'вещ_таб');
CHAR_ARRAY_TYPE     : ('сим' WS? 'таб' | 'сим_таб');
STRING_ARRAY_TYPE   : ('лит' WS? 'таб' | 'лит_таб');
BOOLEAN_ARRAY_TYPE  : ('лог' WS? 'таб' | 'лог_таб');

TRUE                : 'да';
FALSE               : 'нет';
POWER               : '**';
GE                  : '>=';
LE                  : '<=';
NE                  : '<>';
PLUS                : '+';
MINUS               : '-';
MUL                 : '*';
DIV                 : '/';
EQ                  : '=';
LT                  : '<';
GT                  : '>';
LPAREN              : '(';
RPAREN              : ')';
LBRACK              : '[';
RBRACK              : ']';
LBRACE              : '{';
RBRACE              : '}';
COMMA               : ',';
COLON               : ':';
SEMICOLON           : ';';
DIV_OP              : 'div';
MOD_OP              : 'mod';

CHAR_LITERAL        : '\'' ( EscapeSequence | ~['\\\r\n] ) '\'' ;
STRING              : '"' ( EscapeSequence | ~["\\\r\n] )*? '"'
                    | '\'' ( EscapeSequence | ~['\\\r\n] )*? '\''
                    ;
REAL                : (DIGIT+ '.' DIGIT* | '.' DIGIT+) ExpFragment?
                    | DIGIT+ ExpFragment
                    ;
INTEGER             : DecInteger | HexInteger ;

ID                  : LETTER (LETTER | DIGIT | '_' | '@')* ;

LINE_COMMENT        : '|' ~[\r\n]* -> channel(HIDDEN);
DOC_COMMENT         : '#' ~[\r\n]* -> channel(HIDDEN);

WS                  : [ \t\r\n]+ -> skip;

fragment DIGIT      : [0-9];
fragment HEX_DIGIT  : [0-9a-fA-F];
fragment LETTER     : [a-zA-Zа-яА-ЯёЁ];
fragment DecInteger : DIGIT+;
fragment HexInteger : '$' HEX_DIGIT+;
fragment ExpFragment: [eE] [+-]? DIGIT+;
fragment EscapeSequence
                    : '\\' [btnfr"'\\]
                    ;
\end{lstlisting}

В листинге~\ref{lst:kafka-click2} представлен парсер грамматики в формате ANTRL.

\begin{lstlisting}[language=sql, caption={Парсер грамматики в формате ANTRL}, label=lst:kafka-click2]
parser grammar KumirParser;

options { tokenVocab=KumirLexer; } // Use tokens from KumirLexer.g4


qualifiedIdentifier
    : ID // Currently simple ID, can be extended for module.member later
    ;

literal
    : INTEGER | REAL | STRING | CHAR_LITERAL | TRUE | FALSE  | NEWLINE_CONST
    ;

expressionList
    : expression (COMMA expression)*
    ;

arrayLiteral
    : LBRACE expressionList? RBRACE
    ;

primaryExpression 
    : literal
    | qualifiedIdentifier
    | RETURN_VALUE
    | LPAREN expression RPAREN 
    | arrayLiteral
    ;

argumentList 
    : expression (COMMA expression)*
    ;

indexList
    : expression (COLON expression)? 
    | expression COMMA expression
    ;

postfixExpression
    : primaryExpression ( LBRACK indexList RBRACK | LPAREN argumentList? RPAREN )*
    ;

unaryExpression
    : op=(PLUS | MINUS | NOT) unaryExpression | postfixExpression
    ;

powerExpression
    : unaryExpression (POWER powerExpression)?
    ;

multiplicativeExpression
    : powerExpression (op=(MUL | DIV | DIV_OP | MOD_OP) powerExpression)*
    ;

additiveExpression
    : multiplicativeExpression (op=(PLUS | MINUS) multiplicativeExpression)*
    ;

relationalExpression
    : additiveExpression (op=(LT | GT | LE | GE) additiveExpression)*
    ;

equalityExpression
    : relationalExpression (op=(EQ | NE) relationalExpression)*
    ;

logicalAndExpression
    : equalityExpression (AND equalityExpression)*
    ;

logicalOrExpression
    : logicalAndExpression (OR logicalAndExpression)*
    ;

expression
    : logicalOrExpression
    ;

typeSpecifier 
    : arrayType
    | basicType TABLE_SUFFIX?
    ;

basicType
    : INTEGER_TYPE | REAL_TYPE | BOOLEAN_TYPE | CHAR_TYPE | STRING_TYPE
    ;

arrayType
    : INTEGER_ARRAY_TYPE | REAL_ARRAY_TYPE | BOOLEAN_ARRAY_TYPE | CHAR_ARRAY_TYPE | STRING_ARRAY_TYPE
    ;

arrayBounds
    : expression COLON expression
    ;

variableDeclarationItem
    : ID (LBRACK arrayBounds (COMMA arrayBounds)* RBRACK)? ( EQ expression )?
    ;

variableList
    : variableDeclarationItem (COMMA variableDeclarationItem)*
    ;

variableDeclaration
    : typeSpecifier variableList
    ;

globalDeclaration
    : typeSpecifier variableList SEMICOLON?
    ;

globalAssignment
    : qualifiedIdentifier ASSIGN (literal | unaryExpression | arrayLiteral) SEMICOLON?
    ;


parameterDeclaration
    : (IN_PARAM | OUT_PARAM | INOUT_PARAM)? typeSpecifier variableList
    ;

parameterList
    : parameterDeclaration (COMMA parameterDeclaration)*
    ;

algorithmNameTokens
    : ~(LPAREN | ALG_BEGIN | SEMICOLON | EOF)+
    ;

algorithmName: ID+ ;

algorithmHeader
    : ALG_HEADER typeSpecifier? algorithmNameTokens (LPAREN parameterList? RPAREN)? SEMICOLON?
    ;

algorithmBody
    : statementSequence
    ;

statementSequence
    : statement*
    ;

lvalue
    : qualifiedIdentifier (LBRACK indexList RBRACK)?
    | RETURN_VALUE
    ;

assignmentStatement
    : lvalue ASSIGN expression
    | expression
    ;

ioArgument
    : expression (COLON expression (COLON expression)?)?
    | NEWLINE_CONST 
    ;

ioArgumentList
    : ioArgument (COMMA ioArgument)*
    ;

ioStatement
    : (OUTPUT) ioArgumentList
    ;

ifStatement
    : IF expression THEN statementSequence (ELSE statementSequence)? FI
    ;

caseBlock
    : CASE expression COLON statementSequence
    ;

switchStatement
    : SWITCH caseBlock+ (ELSE statementSequence)? FI
    ;

endLoopCondition
    : ENDLOOP_COND expression
    ;

loopSpecifier
    : FOR ID FROM expression TO expression (STEP expression)?
    | WHILE expression
    | expression TIMES
    ;

loopStatement
    : LOOP loopSpecifier? statementSequence (ENDLOOP | endLoopCondition)
    ;

exitStatement
    : EXIT
    ;

statement
    : variableDeclaration SEMICOLON?
    | assignmentStatement SEMICOLON?
    | ioStatement SEMICOLON?
    | ifStatement SEMICOLON?
    | switchStatement SEMICOLON?
    | loopStatement SEMICOLON?
    | exitStatement SEMICOLON?
    | SEMICOLON // Allow empty statements
    ;

algorithmDefinition
    : algorithmHeader (variableDeclaration)*
      ALG_BEGIN
      algorithmBody
      ALG_END algorithmName? SEMICOLON?
    ;

programItem
    : globalDeclaration
    | globalAssignment
    ;

program
    : programItem* algorithmDefinition+ EOF
    ;
\end{lstlisting}

\newpage