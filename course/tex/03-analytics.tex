\section{Аналитический раздел}

Компилятор~---~это программа, которая считывает текст программы, написанной на одном языке~---~исходном, и транслирует (переводит) его в эквивалентный текст на другом языке~---~целевом. Одна из важных ролей компилятора состоит в сообщении об ошибках в исходной программе, обнаруженных в процессе трансляции~\cite{aho2003}.

\subsection{Структура компилятора}

Конструктивно компилятор состоит из~\cite{vladimirtov2004struct,grune2012modern}:
\begin{itemize}
    \item фронтенда (compiler frontend), который занимается построением промежуточного представления из исходного кода и состоит из:
    \begin{itemize}
        \item препроцессора; 
        \item лексического, синтаксического и семантического анализаторов;
        \item генератора промежуточного представления;
    \end{itemize}
    \item мидленд (middle-end), включащий в себя различные оптимизации;
    \item бэкенда (compiler backend), который занимается кодогенерацией.
\end{itemize}

На рисунке~\ref{img:1.png} представлена схема концептуальной структуры компилятора.
% TODO: перерисовать руками
\img{1\textwidth}{1.png}{Концептуальная структура компилятора}

Рассмотрим работу компилятора по фазам~\cite{serebrykov2001}. Обобщенная структура компилятора и основные фазы компиляции показаны на рисунке~\ref{img:2.png}.

% TODO: добавить препроцессор на схему и перерисовать
\img{0.8\textwidth}{2.png}{Обобщенная структура и фазы компиляции}

\subsubsection{Препроцессор}

Иногда сборка поручается программе, который выполняет предварительную обработку перед фазой фронтенда компилятора. 

Препроцессор может~\cite{aho2003,vladimirtov2004struct}:
\begin{enumerate}
    \item раскрывать макросы в инструкции исходного языка;
    \item обрабатывать включение файлов;
    \item обрабатывать языковые расширения.
\end{enumerate}

\subsection{Лексический анализ}

На фазе лексического анализа входная программа, представляющая собой поток литер, разбивается на лексемы~---~слова в соответствии с определениями языка. Основными формализмами, лежащими в основе реализации лексических анализаторов, являются конечные автоматы и регулярные выражения~\cite{serebrykov2001}.

Лексический анализатор может работать в двух основных режимах~\cite{serebrykov2001}:
\begin{enumerate}
    \item как подпрограмма, вызываемая синтаксическим анализатором для получения очередной лексемы;
    \item как полный проход, результатом которого является файл лексем.
\end{enumerate}

В процессе выделения лексем лексический анализатор может~\cite{serebrykov2001}:
\begin{itemize}
    \item cамостоятельно строить таблицы объектов (идентификаторов, строк, чисел и т.д.);
    \item выдавать значения для каждой лексемы при обращении к ней, в этом случае таблицы объектов строятся на последующих фазах (например, при синтаксическом анализе).
\end{itemize}

На этапе лексического анализа обнаруживаются простейшие ошибки~\cite{serebrykov2001}:
\begin{itemize}
    \item недопустимые символы;
    \item неправильная запись чисел;
    \item ошибки в идентификаторах.
\end{itemize}




\newpage

