\documentclass[a4paper,14pt]{extarticle}

\usepackage{cmap} % Улучшенный поиск русских слов в полученном pdf-файле
\usepackage[T2A]{fontenc} % Поддержка русских букв
\usepackage[utf8]{inputenc} % Кодировка utf8
\usepackage[english,russian]{babel} % Языки: русский, английский

\usepackage[14pt]{extsizes} % Задание 14-размера шрифта
\usepackage[left=3cm,right=1cm,top=2cm,bottom=2cm]{geometry} % Задание геометрии листа

\usepackage[unicode,pdftex]{hyperref} % Ссылки в pdf
\hypersetup{hidelinks}

\usepackage{setspace}
\onehalfspacing % Полуторный интервал

\frenchspacing
\usepackage{indentfirst} % Красная строка

% Настройка нумерации объектов
\counterwithin{figure}{section}
\counterwithin{table}{section}
%\numberwithin{equation}{section}


\usepackage{enumitem} % Настройка оформления списков
\setlist{nosep} 
%\setlist[enumerate,1]{label={\arabic*)}}
\setlist[itemize]{left=0.49cm}
\setlist[enumerate]{left=0.49cm}

\renewcommand{\labelitemi}{---}
\renewcommand{\labelitemii}{---}

\usepackage{titlesec} % Оформление заголовков

\titleformat{\section}[block]{\hspace{\parindent}\large\bfseries}{\thesection}{0.5em}{\large\bfseries\raggedright}
\titleformat{\subsection}[block]{\hspace{\parindent}\normalsize\bfseries}{\thesubsection}{0.5em}{\normalsize\bfseries\raggedright}
\titleformat{\subsubsection}[block]{\hspace{\parindent}\normalsize\bfseries}{\thesubsubsection}{0.5em}{\normalsize\bfseries\raggedright}

\titleformat{name=\section,numberless}[block]
{\bfseries\large\centering}
{}
{0em}
{}

\titlespacing{\section}{12.5mm}{10pt}{10pt}
\titlespacing{\subsection}{12.5mm}{10pt}{10pt}
\titlespacing{\subsubsection}{12.5mm}{10pt}{10pt}

% Математические пакеты
\usepackage{amsmath} 
\usepackage{amssymb}

\usepackage{caption} % Подпись картинок и таблиц
\captionsetup{labelsep=endash} % Разделитель между номером и текстом краткое тире и пробел
\captionsetup[figure]{name={Рисунок}} % Изменяет имя для всех фигур на "Рисунок"
\captionsetup[table]{justification=centering, singlelinecheck=false}
\captionsetup[lstlisting]{justification=raggedright, singlelinecheck=false}

% дополнительно к таблицам
\usepackage{makecell} % удобный перенос строки в таблице
\usepackage{longtable} % многостраничные таблицы
\usepackage{multirow} % объединение строк и столбцов


% Вставка рисунков
\usepackage{graphicx} 

\newcommand{\img}[3] {
	\begin{figure}[H]
		\center{\includegraphics[height=#1]{img/#2}}
		\caption{#3}
		\label{img:#2}
	\end{figure}
}

\usepackage{csvsimple} % генерация и фильтрация таблиц из csv файлов
\usepackage{float} % настройка флоат-объектов

%\usepackage[shortcuts]{extdash}


% Оформление листингов
\usepackage{listings}
\usepackage{xcolor} % Добавление цветов 

\lstdefinestyle{mystyle}{ % Опеределение стиля 
	backgroundcolor=\color{white},
	basicstyle=\footnotesize\ttfamily,
	keywordstyle=\color{blue},
	stringstyle=\color{red},
	commentstyle=\color{gray},
	numbers=left,
	numberstyle=\tiny,
	stepnumber=1,
	numbersep=5pt,
	frame=single,
	tabsize=4,
	captionpos=t,
	breaklines=true,
	breakatwhitespace=true,
	xleftmargin=10pt,
	extendedchars=\true
}
\lstset{extendedchars=true, texcl=true}
\lstset{style=mystyle}

% Настройка списка литературы
\addto\captionsrussian{\renewcommand{\refname}{СПИСОК ИСПОЛЬЗОВАННЫХ ИСТОЧНИКОВ}}

\makeatletter
\def\@biblabel#1{#1. }
\makeatother

% вставка многостраничных pdf-документов
\usepackage{pdfpages}

% настройка приложений
\usepackage[titletoc, title]{appendix} %добавление приложений в оглавление

\usepackage[normalem]{ulem}

\usepackage{array}
\newenvironment{signstabular}[1][1]{
	\renewcommand*{\arraystretch}{#1}
	\tabular
}{
	\endtabular
}

%\lstset{
%	literate=
%	{а}{{\selectfont\char224}}1
%	{б}{{\selectfont\char225}}1
%	{в}{{\selectfont\char226}}1
%	{г}{{\selectfont\char227}}1
%	{д}{{\selectfont\char228}}1
%	{е}{{\selectfont\char229}}1
%	{ё}{{\"e}}1
%	{ж}{{\selectfont\char230}}1
%	{з}{{\selectfont\char231}}1
%	{и}{{\selectfont\char232}}1
%	{й}{{\selectfont\char233}}1
%	{к}{{\selectfont\char234}}1
%	{л}{{\selectfont\char235}}1
%	{м}{{\selectfont\char236}}1
%	{н}{{\selectfont\char237}}1
%	{о}{{\selectfont\char238}}1
%	{п}{{\selectfont\char239}}1
%	{р}{{\selectfont\char240}}1
%	{с}{{\selectfont\char241}}1
%	{т}{{\selectfont\char242}}1
%	{у}{{\selectfont\char243}}1
%	{ф}{{\selectfont\char244}}1
%	{х}{{\selectfont\char245}}1
%	{ц}{{\selectfont\char246}}1
%	{ч}{{\selectfont\char247}}1
%	{ш}{{\selectfont\char248}}1
%	{щ}{{\selectfont\char249}}1
%	{ъ}{{\selectfont\char250}}1
%	{ы}{{\selectfont\char251}}1
%	{ь}{{\selectfont\char252}}1
%	{э}{{\selectfont\char253}}1
%	{ю}{{\selectfont\char254}}1
%	{я}{{\selectfont\char255}}1
%	{А}{{\selectfont\char192}}1
%	{Б}{{\selectfont\char193}}1
%	{В}{{\selectfont\char194}}1
%	{Г}{{\selectfont\char195}}1
%	{Д}{{\selectfont\char196}}1
%	{Е}{{\selectfont\char197}}1
%	{Ё}{{\"E}}1
%	{Ж}{{\selectfont\char198}}1
%	{З}{{\selectfont\char199}}1
%	{И}{{\selectfont\char200}}1
%	{Й}{{\selectfont\char201}}1
%	{К}{{\selectfont\char202}}1
%	{Л}{{\selectfont\char203}}1
%	{М}{{\selectfont\char204}}1
%	{Н}{{\selectfont\char205}}1
%	{О}{{\selectfont\char206}}1
%	{П}{{\selectfont\char207}}1
%	{Р}{{\selectfont\char208}}1
%	{С}{{\selectfont\char209}}1
%	{Т}{{\selectfont\char210}}1
%	{У}{{\selectfont\char211}}1
%	{Ф}{{\selectfont\char212}}1
%	{Х}{{\selectfont\char213}}1
%	{Ц}{{\selectfont\char214}}1
%	{Ч}{{\selectfont\char215}}1
%	{Ш}{{\selectfont\char216}}1
%	{Щ}{{\selectfont\char217}}1
%	{Ъ}{{\selectfont\char218}}1
%	{Ы}{{\selectfont\char219}}1
%	{Ь}{{\selectfont\char220}}1
%	{Э}{{\selectfont\char221}}1
%	{Ю}{{\selectfont\char222}}1
%	{Я}{{\selectfont\char223}}1
%}

\usepackage{framed}