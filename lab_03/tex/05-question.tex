\section{Контрольные вопросы}

\subsection{Как может быть определён формальный язык?}

Формальный язык может быть определён, например:
\begin{enumerate}
	\item простым перечислением слов, входящих в данный язык. Этот способ, в основном, применим для определения конечных языков и языков простой структуры;
	\item словами, порождёнными некоторой формальной грамматикой;
	\item словами, порождёнными регулярным выражением;
	\item словами, распознаваемыми некоторым конечным автоматом;
	\item словами, порождёнными БНФ-конструкцией.
\end{enumerate}

\subsection{Какими характеристиками определяется грамматика?}

Грамматика определяется следующими характеристиками:
\begin{enumerate}
	\item $\Sigma$~---~набор (алфавит) терминальных символов;
	\item $N$~---~набор (алфавит) нетерминальных символов;
	\item $P$~---~набор правил вида: «левая часть» $\rightarrow$ «правая часть», где:
		\begin{itemize}
			\item «левая часть»~---~непустая последовательность терминалов и нетерминалов, содержащая хотя бы один нетерминал;
			\item «правая часть»~---~любая последовательность терминалов и нетерминалов;
		\end{itemize}
	\item $S$~---~стартовый (или начальный) символ грамматики из набора нетерминалов.
\end{enumerate}

\subsection{Дайте описания грамматик по иерархии Хомского.}

Грамматика с фразовой структурой $G$~---~это алгебраическая структура, упорядоченная четвёрка $(V_T, V_N, P, S)$, где:
\begin{itemize}
	\item $V_T$~---~алфавит (множество) терминальных символов;
	\item $V_N$~---~алфавит (множество) нетерминальных символов;
	\item $V = V_T \cup V_N$~---~словарь $G$, причём $V_T \cap V_N = \varnothing$;
	\item $P$~---~конечное множество продукций (правил) грамматики, $P \subseteq V^+ \times V^*$;
	\item $S$~---~начальный символ (источник).
\end{itemize}

Здесь $V^{*}$~---~множество всех строк над алфавитом $V$, а $V^{+}$~---~множество непустых строк над алфавитом $V$.

По иерархии Хомского, грамматики делятся на 4 типа, каждый последующий является более ограниченным подмножеством предыдущего (но и легче поддающимся анализу).
\begin{enumerate}
	\item неограниченные грамматики — возможны любые правила;
	\item контекстно-зависимые грамматики — левая часть может содержать один нетерминал, окруженный «контекстом» (последовательности символов, в том же виде присутствующие в правой части); сам нетерминал заменяется непустой последовательностью символов в правой части;
	\item контекстно-свободные грамматики — левая часть состоит из одного нетерминала;
	\item регулярные грамматики — более простые, эквивалентны конечным автоматам.
\end{enumerate}

\subsubsection{Неограниченные грамматики}
Это все без исключения формальные грамматики. Правила можно записать в виде: $\alpha \rightarrow \beta$, 
где $\alpha \in V^{+}$~---~любая непустая цепочка, содержащая хотя бы один нетерминальный символ, 
а $\beta \in V^{*}$~---~любая цепочка символов из алфавита.

\subsubsection{Контекстно-зависимые грамматики}
К этому типу относятся контекстно-зависимые (КЗ) грамматики и неукорачивающие грамматики. Для грамматики 
$G(V_{T},V_{N},P,S)$, $V=V_{T}\cup V_{N}$ все правила имеют вид:
\begin{itemize}
	\item $\alpha A\beta \rightarrow \alpha \gamma \beta$, где $\alpha ,\beta \in V^{*}$, $\gamma \in V^{+}$, $A\in V_{N}$. Такие грамматики относят к контекстно-зависимым.
	\item $\alpha \rightarrow \beta$, где $\alpha ,\beta \in V^{+}$, $1\leq |\alpha |\leq |\beta |$. Такие грамматики относят к неукорачивающим.
\end{itemize}

\subsubsection{Контекстно-свободные грамматики}
Для грамматики $G(V_{T},V_{N},P,S)$, $V=V_{T}\cup V_{N}$ все правила имеют вид: $A\rightarrow \beta$, где $\beta \in V^{+}$ (для неукорачивающих КС-грамматик) или $\beta \in V^{*}$ (для укорачивающих), 
$A\in V_{N}$. То есть грамматика допускает появление в левой части правила только нетерминального символа.

\subsubsection{Регулярные грамматики}
К третьему типу относятся регулярные грамматики (автоматные)~---~самые простые из формальных грамматик. Они являются контекстно-свободными, но с ограниченными возможностями.

Все регулярные грамматики могут быть разделены на два эквивалентных класса, которые для грамматики вида III будут иметь правила следующего вида:
\begin{itemize}
	\item $A\rightarrow B\gamma$ или $A\rightarrow \gamma$, где $\gamma \in V_{T}^{*}$, $A,B\in V_{N}$ (для леволинейных грамматик).
	\item $A\rightarrow \gamma B$ или $A\rightarrow \gamma$, где $\gamma \in V_{T}^{*}$, $A,B\in V_{N}$ (для праволинейных грамматик).
\end{itemize}

\subsection{Какие абстрактные устройства используются для разбора грамматик?}
\begin{enumerate}
	\item Для разбора слов из регулярных языков подходят формальные автоматы самого простого устройства, т. н. конечные автоматы. Их функция перехода задаёт только смену состояний и, возможно, сдвиг (чтение) входного символа.
	\item Для разбора слова из контекстно-свободных языков в автомат приходится добавлять «магазинную ленту» или «стек», в который при каждом переходе записывается цепочка на основе соответствующего алфавита магазина. Такие автоматы называют «магазинные автоматы».
	\item Для контекстно-зависимых языков разработаны ещё более сложные линейно-ограниченные автоматы, а для языков общего вида — машина Тьюринга.
\end{enumerate}

\subsection{Оцените временную и емкостную сложность предложенного вам алгоритма.}

Временная сложность~---~$O(|P|^2)$, где $P$~---~конечное множество продукций (правил) грамматики.

Ёмкостная сложность~---~$O(|P|)$, где $P$~---~конечное множество продукций (правил) грамматики.

\newpage