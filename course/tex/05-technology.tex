\section{Технологический раздел}

\subsection{Выбор средств программной реализации}

В качестве языка реализации компилятора выбран Go. Это решение обусловлено следующими факторами.

\begin{itemize}
    \item Кросс-платформенность: Скомпилированный компилятор может выполняться на различных операционных системах и архитектурах.
    
    \item Интеграция с LLVM: Существуют готовые библиотеки для генерации LLVM IR-кода из программ на Go.
    
    \item Поддержка инструментария: Генератор анализаторов ANTLR предоставляет возможность генерации кода на языке Go.
\end{itemize}

\subsection{Основные компоненты программы}

В результате работы ANTLR были сгенерированы интерфейсы BaseVisitor и BaseListener, файлы с данными для интерпретатора ANTLR и файлы с токенами и реализации анализаторов.

Был реализован интерфейс BaseVisitor, т.к. он предоставляет контролируемый обход с явным указанием порядка посещения узлов через специализированные методы вида \texttt{VisitXXX}. Пример реализации такого метода представлен в листинге~\ref{lst:visitor-example}.

\begin{lstlisting}[language=go, caption={Пример реализации метода VisitBasicType для определения типа скалярной переменной}, label=lst:visitor-example]
func (v *IRVisitor) VisitBasicType(ctx *parser.BasicTypeContext) interface{} {
	if ctx.INTEGER_TYPE() != nil {
		return types.I32
	} else if ctx.REAL_TYPE() != nil {
		return types.Double
	} else if ctx.BOOLEAN_TYPE() != nil {
		return types.I1
	} else if ctx.CHAR_TYPE() != nil {
		return types.I8
	} else if ctx.STRING_TYPE() != nil {
		return types.NewPointer(types.I8)
	}
	v.Errors = append(v.Errors, fmt.Errorf("unknown basic type: %s", ctx.GetText()))
	return types.I32
}
\end{lstlisting}

\subsubsection*{Порядок функций}

В КуМир нет четкой определенной функции \texttt{main}. В ее роли выступает первая встреченная функция. Функции, определенные за ней будут являться обычными функциями, которые можно будет вызвать из главной. 

Однако, в \texttt{LLVM IR} выполнение должно начинать с функции \texttt{main}. Для этого используется код, представленный в листинге~\ref{lst:main}.

\begin{lstlisting}[language=go, caption={Определение main-функции}, label=lst:main]
func (v *IRVisitor) declareAlgorithmPrototype(ctx *parser.AlgorithmHeaderContext) {
    ... 
	if v.mainOriginalName == "" {
		v.mainOriginalName = llvmFuncName
		llvmFuncName = "main"
	}
    ...
}
\end{lstlisting}

В связи с тем, что объявление используемых функций узнаются позже, чем сгенерируется LLVM-представление происходит ситуация, что мы не знаем о других функциях в программе, а объявлять их раньше нельзя, т.к. они будут считаться главными. Для решения этой проблемы было принято решение делать предварительный обход по всем заголовкам функций и создавать их прототипы. Код обхода представлен в листинге~\ref{lst:declare}

\begin{lstlisting}[language=go, caption={Обход по заголовкам функций}, label=lst:declare]
func (v *IRVisitor) VisitProgram(ctx *parser.ProgramContext) interface{} {
    ...
	for _, item := range ctx.AllAlgorithmDefinition() {
		v.declareAlgorithmPrototype(item.AlgorithmHeader().
            (*parser.AlgorithmHeaderContext))
	}
	for _, item := range ctx.AllAlgorithmDefinition() {
		v.Visit(item)
	}
    ...
}
\end{lstlisting}

\subsubsection*{Кириллические имена функций и переменных}

Синтаксис языка КуМир представлен на русском языке, он позволяет в качестве идентификаторов использовать кириллические имена функций и переменных, но \texttt{LLVM IR} такого не допускает. Поэтому при каждом получении идентификатора он проходит обработку представленную в листинге~\ref{lst:normal}.

\begin{lstlisting}[language=go, caption={Подготовка имени к использованию в дальнейшем коде}, label=lst:normal]
func containsNonASCII(s string) bool {
	for _, r := range s {
		if r > unicode.MaxASCII {
			return true
		}
	}
	return false
}

func NormalizeFunctionName(name string) string {
	transliterated := translit.EncodeToICAO(name)
	if containsNonASCII(name) {
		transliterated += "_rus"
	}

	var sb strings.Builder
	for i, r := range transliterated {
		if (r >= 'a' && r <= 'z') || (r >= 'A' && r <= 'Z') || (r >= '0' && r <= '9') || r == '_' {
			sb.WriteRune(r)
		} else if i == 0 && unicode.IsDigit(r) {
			sb.WriteRune('_')
			sb.WriteRune(r)
		} else {
			sb.WriteRune('_')
		}
	}
	return sb.String()
}
\end{lstlisting}

Код, представленный в этом листинге сначала транлитизирует кириллицу в латиницу, а после добавляет суффикс <<\_rus>> для избежания конфликтов в ситуации наименования функций <<Факт>> и <<Fact>>. Если идентификатор изначально был на латинице, то с ним ничего не происходит.

\subsubsection*{Статическая типизация}

Язык КуМир строго типизирован так же, однако в нем так же, как и в $C$ есть неявное приведение типов, а точнее расширение с типа \texttt{цел} до типа \texttt{вещ}, что равносильно расширению с \texttt{int} до \texttt{double} в $C$. Для реализации неявного расширения типов была реализована функция, представленная в листинге~\ref{lst:cast}. 

\begin{lstlisting}[language=go, caption={Пример реализации метода VisitBasicType для определения типа скалярной переменной}, label=lst:cast]
func (v *IRVisitor) castToMatch(lhs, rhs value.Value) (value.Value, value.Value) {
	lt, rt := lhs.Type(), rhs.Type()
	if lt.Equal(rt) {
		return lhs, rhs
	}
	if _, ok := lt.(*types.IntType); ok {
		if _, isFloat := rt.(*types.FloatType); isFloat {
			lhs = v.currentBlock.NewSIToFP(lhs, rt)
			return lhs, rhs
		}
	}
	if _, ok := rt.(*types.IntType); ok {
		if _, isFloat := lt.(*types.FloatType); isFloat {
			rhs = v.currentBlock.NewSIToFP(rhs, lt)
			return lhs, rhs
		}
	}
	return lhs, rhs
}
\end{lstlisting}

\subsubsection*{Базовые функции языка}

Были так же реализованы базовые функции языка, а именно:
\begin{itemize}
    \item арифметические операции;
    \item логические операции;
    \item операции сравнения;
    \item условные конструкции;
    \item циклы;
    \item досрочный выход из цикла (break);
    \item функции и процедуры;
    \item функция вывода в стандартный поток ввода-вывода.
\end{itemize}

\subsection{Тестирование}

Было проведено тестирование работы компилятора для базовых конструкций КуМир в соответствии с грамматикой. Примеры программ для тестирования представлены в приложении~\ref{appendix:b}.

В эти примеры вошли:
\begin{itemize}
    \item выводы различных величин;
    \item поиск максимума с помощью вложенных условных конструкций;
    \item 7 видов циклов, в том числе бесконечный и вложенный;
    \item программа с процедурой;
    \item программа с оператором выбора (switch);
    \item программа с выводом различных арифметических операций;
    \item рекурсивное вычисление факториала.
\end{itemize}

\subsection{Пример работы программы}

Для запуска работы программы достаточно выполнить одну из команд, представленных в листинге~\ref{lst:start}.

\begin{lstlisting}[language=go, caption={Пример запуска реализованного компилятора}, label=lst:start]
Если запускать через go:
go run ./cmd/compiler/main.go -i <путь_до_файла> 
Если программа скомпилирована в бинарник:
main -i <путь_до_файла> -o <путь_до_файла> 
\end{lstlisting}

Кроме того, реализован следующий \texttt{CLI}-интерфейс:
\begin{itemize}
    \item флаг \texttt{-i} указывает путь до файла с исходным кодом на языке КуМир, по умолчанию равен <<./example/2+2.kum>>;
    \item флаг \texttt{-o} указывает на путь скомпилированного исполняемого файла, по умолчанию равен <<./example/out>>;
    \item флаг \texttt{-d} позволяет сгенерировать AST-дерево для указанного файла с исходным кодом, по умолчанию равен \texttt{False}.
\end{itemize}

Примеры программ на КуМир и соответствующий им результат работы компилятора на LLVM IR представлены в Приложении~\ref{appendix:c}.

К этим примерам относятся:
\begin{itemize}
    \item рекурсивное вычисление N-го числа Фибоначчи;
    \item циклическое вычисление N-го числа Фибоначчи;
    \item реверс массива.
\end{itemize}

\newpage
	