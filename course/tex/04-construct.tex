\section{Конструкторский раздел}

\subsection{Концептуальная модель}

Концептуальная модель разрабатываемого компилятора в нотации IDEF0 представлена на рисунке~\ref{img:1-5.pdf}.

\img{1\textwidth}{1-5.pdf}{Концептуальная модель разрабатываемого компилятора в нотации IDEF0}

\subsection{Язык КуМир}

КуМир (Комплект Учебных МИРов) - система программирования, предназначенная для поддержки начальных курсов информатики и программирования в средней и высшей школе.

% расписать откуда кто придумал и где используется

Грамматика языка представлена в приложении~\ref{appendix:a}.

\subsection{Лексический и синтаксический анализ}
В данной работе для генерации лексического и синтаксического анализаторов используется инструмент \textsc{ANTLR4}. На вход системы подаётся формальное описание грамматики языка в формате, поддерживаемом \textsc{ANTLR4}.

Процесс генерации включает создание:
\begin{itemize}
    \item Классов лексера (Lexer) и парсера (Parser)
    \item Вспомогательных классов и файлов поддержки
    \item Шаблонов классов для обхода синтаксического дерева
\end{itemize}

Анализ выполняется последовательно:
\begin{enumerate}
    \item Лексер преобразует входной поток символов (исходный код) в поток токенов
    \item Парсер обрабатывает поток токенов, формируя дерево разбора (parse tree)
\end{enumerate}
Ошибки, обнаруженные на этапах лексического и синтаксического анализа, выводятся в стандартный поток вывода.

\subsection{Семантический анализ}
Для обхода абстрактного синтаксического дерева (АСТ) доступны две стратегии:
\begin{itemize}
    \item \textbf{Listener:} Реализует автоматический обход в глубину, активируя обработчики при входе в узел и выходе из него
    \item \textbf{Visitor:} Предоставляет контролируемый обход с явным указанием порядка посещения узлов через специализированные методы
\end{itemize}

В представленной реализации используется паттерн \textsc{Visitor}, обеспечивающий:
\begin{itemize}
    \item Гибкое управление порядком обхода
    \item Возможность выборочной обработки узлов
    \item Реализацию специализированных методов посещения для каждого типа узла
\end{itemize}
Обход начинается с корневого узла, соответствующего точке входа программы.

\newpage