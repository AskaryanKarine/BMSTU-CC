\section{Технологический раздел}

\subsection{Выбор средств программной реализации}

В качестве языка реализации компилятора выбран Go. Это решение обусловлено следующими факторами.

\begin{itemize}
    \item Кросс-платформенность: Скомпилированный компилятор может выполняться на различных операционных системах и архитектурах.
    
    \item Интеграция с LLVM: Существуют готовые библиотеки для генерации LLVM IR-кода из программ на Go.
    
    \item Поддержка инструментария: Генератор анализаторов ANTLR предоставляет возможность генерации кода на языке Go.
\end{itemize}

\subsection{Основные компоненты программы}

В результате работы ANTLR были сгенерированы интерфейсы BaseVisitor и BaseListener, файлы с данными для интерпретатора ANTLR и файлы с токенами и реализации анализаторов.

Был реализован интерфейс BaseVisitor, т.к. он предоставляет контролируемый обход с явным указанием порядка посещения узлов через специализированные методы вида \texttt{VisitXXX}. Пример реализации такого метода представлен в листинге~\ref{lst:visitor-example}.

\begin{lstlisting}[language=go, caption={Пример реализации метода BaseVisitor}, label=lst:visitor-example]
TODO
\end{lstlisting}

% TODO: расписать когда добью реализацию
\subsubsection*{Статическая типизация}

TODO

\subsubsection*{Базовые функции языка}

TODO

\subsection{Тестирование}

Было проведено тестирование работы компилятора для базовых конструкций КуМир в соответствии с грамматикой. Примеры программ для тестирования представлены в приложении~\ref{appendix:b}.

\subsection{Пример работы программы}

Примеры программ на КуМир и соответствующий им результат работы компилятора на LLVM IR представлены в Приложении~\ref{appendix:c}.

%  нужно ли писать про флаги и вывод в dot формате? - можно описать cli

\newpage
	