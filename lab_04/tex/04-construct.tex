\section{Практическая часть}

\subsection{Результат выполнения работы}

В листинге~\ref{lst:input} представлены входные данные. На рисунке~\ref{img:ast.png} --- построенное AST-дерево.

\begin{lstlisting}[language=Go, caption={Входная программа}, label=lst:input]
a < b + c
\end{lstlisting}

\img{0.5\textwidth}{ast.png}{}

\subsection{Код программы}

В листингах~\ref{lst1}~---~\ref{lst2} приведен код программы на языке Go.

\begin{lstlisting}[language=Go, caption={Код модуля \textit{lexer}}, label=lst1]
package lexer

import "unicode"

type Lexer struct {
	input   string
	pos     int
	line    int
	column  int
	start   int
	startLn int
	startCl int
}

func NewLexer(input string) *Lexer {
	return &Lexer{
		input:  input,
		line:   1,
		column: 1,
	}
}

func (l *Lexer) NextToken() Token {
	l.skipWhitespace()

	l.start = l.pos
	l.startLn = l.line
	l.startCl = l.column

	if l.pos >= len(l.input) {
		return Token{Type: TokenEOF}
	}

	if typ, ok := l.tryOperator(); ok {
		return typ
	}

	ch := l.input[l.pos]

	if unicode.IsLetter(rune(ch)) {
		return l.readIdentifier()
	}

	if unicode.IsDigit(rune(ch)) {
		return l.readNumber()
	}

	l.pos++
	l.column++
	return Token{Type: TokenERROR, Literal: string(ch)}
}

func (l *Lexer) tryOperator() (Token, bool) {
	if l.pos+1 < len(l.input) {
		sub := l.input[l.pos : l.pos+2]
		if typ, ok := operators[sub]; ok {
			tok := Token{
				Type:    typ,
				Literal: sub,
			}
			l.pos += 2
			l.column += 2
			return tok, true
		}
	}

	sub := l.input[l.pos : l.pos+1]
	if typ, ok := operators[sub]; ok {
		tok := Token{
			Type:    typ,
			Literal: sub,
		}
		l.pos++
		l.column++
		return tok, true
	}

	return Token{}, false
}

func (l *Lexer) skipWhitespace() {
	for l.pos < len(l.input) {
		ch := l.input[l.pos]
		if ch == '\n' {
			l.line++
			l.column = 1
			l.pos++
			continue
		}
		if unicode.IsSpace(rune(ch)) {
			l.pos++
			l.column++
			continue
		}
		break
	}
}

func (l *Lexer) readIdentifier() Token {
	start := l.pos
	for l.pos < len(l.input) {
		ch := rune(l.input[l.pos])
		if !unicode.IsLetter(ch) && !unicode.IsDigit(ch) && ch != '_' {
			break
		}
		l.pos++
		l.column++
	}

	literal := l.input[start:l.pos]

	if typ, ok := keywords[literal]; ok {
		return Token{Type: typ, Literal: literal}
	}
	return Token{Type: TokenIDENT, Literal: literal}
}

func (l *Lexer) readNumber() Token {
	start := l.pos
	for l.pos < len(l.input) {
		ch := rune(l.input[l.pos])
		if !unicode.IsDigit(ch) {
			break
		}
		l.pos++
		l.column++
	}

	literal := l.input[start:l.pos]
	return Token{Type: TokenNUMBER, Literal: literal}
}
\end{lstlisting}

\begin{lstlisting}[language=Go, caption={Код модуля \textit{lexer}}]
package lexer

const (
	TokenEOF = iota
	TokenERROR
	TokenIDENT
	TokenNUMBER
	TokenBEGIN
	TokenEND
	TokenSEMICOLON
	TokenASSIGN
	TokenEQ
	TokenPLUS
	TokenMINUS
	TokenMULT
	TokenDIV
	TokenLT
	TokenLE
	TokenGT
	TokenGE
	TokenNE
	TokenLPAREN
	TokenRPAREN
)

var (
	keywords = map[string]int{
		"begin": TokenBEGIN,
		"end":   TokenEND,
	}

	operators = map[string]int{
		";":  TokenSEMICOLON,
		"=":  TokenASSIGN,
		"==": TokenEQ,
		"+":  TokenPLUS,
		"-":  TokenMINUS,
		"*":  TokenMULT,
		"/":  TokenDIV,
		"(":  TokenLPAREN,
		")":  TokenRPAREN,
		"<":  TokenLT,
		"<=": TokenLE,
		">":  TokenGT,
		">=": TokenGE,
		"<>": TokenNE,
	}
)

type Token struct {
	Type    int
	Literal string
}

func (l *Lexer) Tokenize() []Token {
	var tokens []Token
	for {
		tok := l.NextToken()
		tokens = append(tokens, tok)
		if tok.Type == TokenEOF || tok.Type == TokenERROR {
			break
		}
	}
	return tokens
}
\end{lstlisting}

\begin{lstlisting}[language=Go, caption={Код модуля \textit{parser}}]
package parser

import "github.com/AskaryanKarine/BMSTU-CC/lab_04/internal/lexer"

type ASTNode struct {
	Type     string
	Value    string
	Children []*ASTNode
	Token    lexer.Token
}

type Parser struct {
	tokens []lexer.Token
	pos    int
}

func NewParser(tokens []lexer.Token) *Parser {
	return &Parser{
		tokens: tokens,
	}
}

func (p *Parser) CurrentToken() lexer.Token {
	if p.pos >= len(p.tokens) {
		return lexer.Token{Type: lexer.TokenEOF}
	}
	return p.tokens[p.pos]
}

func (p *Parser) advance() {
	p.pos++
}

func (p *Parser) match(tokenType int) bool {
	if p.CurrentToken().Type == tokenType {
		p.advance()
		return true
	}
	return false
}

func (p *Parser) expect(tokenType int) (lexer.Token, bool) {
	tok := p.CurrentToken()
	if tok.Type == tokenType {
		p.advance()
		return tok, true
	}
	return tok, false
}

// Правила грамматики
func (p *Parser) parseFactor() (*ASTNode, bool) {
	tok := p.CurrentToken()
	switch tok.Type {
	case lexer.TokenIDENT:
		p.advance()
		return &ASTNode{
			Type:  "Factor",
			Value: tok.Literal,
			Token: tok,
		}, true
	case lexer.TokenNUMBER:
		p.advance()
		return &ASTNode{
			Type:  "Factor",
			Value: tok.Literal,
			Token: tok,
		}, true
	case lexer.TokenLPAREN:
		p.advance()
		expr, ok := p.parseArithExpr()
		if !ok {
			return nil, false
		}
		if _, ok := p.expect(lexer.TokenRPAREN); !ok {
			return nil, false
		}
		return &ASTNode{
			Type:     "Factor",
			Children: []*ASTNode{expr},
			Token:    tok,
		}, true
	default:
		return nil, false
	}
}

func (p *Parser) parseTerm() (*ASTNode, bool) {
	left, ok := p.parseFactor()
	if !ok {
		return nil, false
	}

	for {
		tok := p.CurrentToken()
		if tok.Type == lexer.TokenMULT || tok.Type == lexer.TokenDIV {
			p.advance()
			right, ok := p.parseFactor()
			if !ok {
				return nil, false
			}
			left = &ASTNode{
				Type: "Term",
				Children: []*ASTNode{
					left,
					{Type: "MulOp", Value: tok.Literal, Token: tok},
					right,
				},
				Token: tok,
			}
		} else {
			break
		}
	}
	return left, true
}

func (p *Parser) parseArithExpr() (*ASTNode, bool) {
	left, ok := p.parseTerm()
	if !ok {
		return nil, false
	}

	for {
		tok := p.CurrentToken()
		if tok.Type == lexer.TokenPLUS || tok.Type == lexer.TokenMINUS {
			p.advance()
			right, ok := p.parseTerm()
			if !ok {
				return nil, false
			}
			left = &ASTNode{
				Type: "ArithExpr",
				Children: []*ASTNode{
					left,
					{Type: "AddOp", Value: tok.Literal, Token: tok},
					right,
				},
				Token: tok,
			}
		} else {
			break
		}
	}
	return left, true
}

func (p *Parser) parseExpression() (*ASTNode, bool) {
	left, ok := p.parseArithExpr()
	if !ok {
		return nil, false
	}

	tok := p.CurrentToken()
	if tok.Type == lexer.TokenLT || tok.Type == lexer.TokenLE ||
		tok.Type == lexer.TokenGT || tok.Type == lexer.TokenGE ||
		tok.Type == lexer.TokenEQ || tok.Type == lexer.TokenNE {
		p.advance()
		right, ok := p.parseArithExpr()
		if !ok {
			return nil, false
		}
		return &ASTNode{
			Type: "Expression",
			Children: []*ASTNode{
				left,
				{Type: "RelOp", Value: tok.Literal, Token: tok},
				right,
			},
			Token: tok,
		}, true
	}
	return &ASTNode{
		Type:     "Expression",
		Children: []*ASTNode{left},
		Token:    tok,
	}, true
}

func (p *Parser) parseProgram() (*ASTNode, bool) {
	expr, ok := p.parseExpression()
	if !ok {
		return nil, false
	}

	if p.CurrentToken().Type != lexer.TokenEOF {
		return nil, false
	}
	return expr, true
}

func (p *Parser) Parse() (*ASTNode, bool) {
	return p.parseProgram()
}

func ASTToRPN(node *ASTNode) []string {
	if node == nil {
		return nil
	}

	var rpn []string

	if len(node.Children) == 3 &&
		(node.Type == "Expression" || node.Type == "ArithExpr" || node.Type == "Term") {
		left := ASTToRPN(node.Children[0])
		right := ASTToRPN(node.Children[2])
		op := ASTToRPN(node.Children[1])

		rpn = append(rpn, left...)
		rpn = append(rpn, right...)
		rpn = append(rpn, op...)
		return rpn
	}

	if node.Type == "Factor" {
		if len(node.Children) == 3 {
			return ASTToRPN(node.Children[1])
		} else if len(node.Children) == 1 {
			return ASTToRPN(node.Children[0])
		}
	}

	if node.Value != "" {
		return []string{node.Value}
	}

	for _, child := range node.Children {
		rpn = append(rpn, ASTToRPN(child)...)
	}

	return rpn
}

func ASTToRPN1(node *ASTNode) ([]string, *ASTNode) {
	if node == nil {
		return nil, nil
	}

	switch node.Type {
	case "Expression", "ArithExpr", "Term":
		var rpn []string
		var rpnNode *ASTNode
		var childrenNodes []*ASTNode

		for _, child := range node.Children {
			childRPN, childNode := ASTToRPN1(child)
			rpn = append(rpn, childRPN...)
			if childNode != nil {
				childrenNodes = append(childrenNodes, childNode)
			}
		}

		// Для RPN дерева создаем узел-оператор с операндами в качестве детей
		if len(childrenNodes) > 0 && node.Value != "" {
			rpnNode = &ASTNode{
				Type:  node.Type + "_RPN",
				Value: node.Value,
			}
			for _, child := range childrenNodes {
				rpnNode.Children = append(rpnNode.Children, child)
			}
		} else if len(childrenNodes) > 0 {
			rpnNode = &ASTNode{
				Type:     node.Type + "_RPN",
				Children: childrenNodes,
			}
		}

		return rpn, rpnNode

	case "Factor", "Identifier", "Number":
		return []string{node.Value}, &ASTNode{
			Type:  node.Type + "_RPN",
			Value: node.Value,
			Token: node.Token,
		}

	case "AddOp", "MulOp", "RelOp":
		return []string{node.Value}, &ASTNode{
			Type:  node.Type + "_RPN",
			Value: node.Value,
			Token: node.Token,
		}

	default:
		return nil, nil
	}
}
\end{lstlisting}

\begin{lstlisting}[language=Go, caption={Код модуля \textit{parser}}]
package parser

import (
	"fmt"
	"strings"
)

func GenerateDOT(node *ASTNode) string {
	var builder strings.Builder
	builder.WriteString("digraph AST {\n")
	builder.WriteString("  node [shape=box, fontname=\"Courier\", fontsize=10];\n")
	builder.WriteString("  edge [fontname=\"Courier\", fontsize=10];\n\n")

	var nodeCounter int
	generateDOTNode(&builder, node, &nodeCounter)

	builder.WriteString("}\n")
	return builder.String()
}

func generateDOTNode(builder *strings.Builder, node *ASTNode, counter *int) int {
	if node == nil {
		return -1
	}

	currentID := *counter
	*counter++

	label := node.Type
	if node.Value != "" {
		label += fmt.Sprintf("\\n%s", node.Value)
	}

	builder.WriteString(fmt.Sprintf("  node%d [label=\"%s\"];\n", currentID, label))

	for _, child := range node.Children {
		childID := generateDOTNode(builder, child, counter)
		if childID >= 0 {
			builder.WriteString(fmt.Sprintf("  node%d -> node%d;\n", currentID, childID))
		}
	}

	return currentID
}
\end{lstlisting}

\begin{lstlisting}[language=Go, caption={Код модуля \textit{fs}}, label=lst2]
package fs

import (
	"bufio"
	"fmt"
	"os"
	"os/exec"
	"strings"
)

func ReadProgramFile(filename string) (string, error) {
	file, err := os.Open(filename)
	if err != nil {
		return "", fmt.Errorf("ошибка открытия файла: %w", err)
	}
	defer file.Close()

	scanner := bufio.NewScanner(file)
	var lines []string
	for scanner.Scan() {
		lines = append(lines, scanner.Text())
	}
	input := strings.Join(lines, "\n")
	return input, nil
}

func SaveDOTToFile(dot, filenameDOT, filenamePNG string) error {
	file, err := os.Create(filenameDOT)
	if err != nil {
		return err
	}
	defer file.Close()

	_, err = file.WriteString(dot)
	if err != nil {
		return err
	}
	cmd := exec.Command("dot", "-Tpng", "-o", filenamePNG, filenameDOT)
	if _, err := cmd.CombinedOutput(); err != nil {
		return err
	}

	return nil
}

\end{lstlisting}


\newpage