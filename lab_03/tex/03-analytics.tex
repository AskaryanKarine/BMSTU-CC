\section{Теоретическая часть}

\textbf{Цель работы:} приобретение практических навыков реализации алгоритма рекурсивного спуска для разбора грамматики и построения синтаксического дерева.

\textbf{Задачи работы:}

\begin{enumerate}
	\item Познакомиться с методом рекурсивного спуска для синтаксического анализа.
	\item Разработать, тестировать и отладить программу построения синтаксического дерева методом рекурсивного спуска в соответствии с предложенным вариантом грамматики.
\end{enumerate}

\subsection{Задание}

\begin{enumerate}
	\item Дополнить грамматику по варианту блоком, состоящим из последовательности операторов присваивания (выбран стиль Алгол-Паскаль).
	\item Для модифицированной грамматики написать программу нисходящего синтаксического анализа с использованием метода рекурсивного спуска.
\end{enumerate}

Блок в стиле Алгол-Паскаль:
   
\begin{framed}
\ttfamily 
\begin{alltt}
<программа> -> <блок>
<блок> -> begin <список операторов> end
<список операторов> -> 
        <оператор> | <список операторов> ; <оператор> 
<оператор> -> <идентификатор> = <выражение>
\end{alltt}
\end{framed}

Грамматика по варианту:

\begin{framed}
\ttfamily 
\begin{alltt}
<выражение> -> 
    <арифм выражение> <операция отношения> <арифм выражение> | 
    <арифм выражение> 

<арифм выражение> -> 
    <арифм выражение> <операция типа сложения> <терм> | 
    <терм> 

<терм> -> 
    <терм> <операция типа умножения> <фактор> | 
    <фактор> 

<фактор> -> 
    <идентификатор> | 
    <константа> | 
    ( <арифм выражение> ) 

<операция отношения> -> < | <= | = | <> | > | >= 
<операция типа сложения> -> + | - 
<операция типа умножения> -> * | / 
\end{alltt}
\end{framed}

Грамматика по варианту после удаления левой рекурсии и добавления блока:

\begin{framed}
\ttfamily 
\begin{alltt}
<программа> -> <блок>
<блок> -> begin <список операторов> end
<список операторов> -> <оператор> ; <оператор_хвост>
<оператор_хвост> -> <оператор> ; <оператор_хвост> | epsilon
<оператор> -> <идентификатор> = <выражение>

<выражение> -> 
    <арифм выражение> <операция отношения> <арифм выражение> | 
    <арифм выражение> 

<арифм выражение> -> 
    <терм> | 
    <арифм выражение> <операция типа сложения> <терм>   

<терм> -> 
    <фактор> | 
    <терм> <операция типа умножения> <фактор>

<фактор> -> 
    <идентификатор> | 
    <константа> | 
    ( <арифм выражение> )

<операция отношения> -> < | <= | = | <> | > | >= 
<операция типа сложения> -> + | - 
<операция типа умножения> -> * | / 
\end{alltt}
\end{framed}

\newpage