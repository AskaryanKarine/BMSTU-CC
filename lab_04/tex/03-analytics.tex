\section{Теоретическая часть}

\textbf{Цель работы:} приобретение практических навыков реализации синтаксически управляемого перевода.

\textbf{Задачи работы:}

\begin{enumerate}
	\item Разработать, тестировать и отладить программу синтаксического анализа в соответствии с предложенным вариантом грамматики.
	\item Включить в программу синтаксического анализ семантические действия для реализации синтаксически управляемого перевода инфиксного выражения в обратную польскую нотацию.
\end{enumerate}

\subsection{Задание}

Реализовать синтаксически управляемый перевод инфиксного выражения в обратную польскую нотацию для грамматики выражений из лабораторной работы №3. Для построения дерева разбора использовать синтаксический анализатор для данной грамматики разработанный в лабораторной работы №3.

Грамматика по варианту для выражений:

\begin{framed}
\ttfamily 
\begin{alltt}
<выражение> -> 
    <арифм выражение> <операция отношения> <арифм выражение> | 
    <арифм выражение> 

<арифм выражение> -> 
    <арифм выражение> <операция типа сложения> <терм> | 
    <терм> 

<терм> -> 
    <терм> <операция типа умножения> <фактор> | 
    <фактор> 

<фактор> -> 
    <идентификатор> | 
    <константа> | 
    ( <арифм выражение> ) 

<операция отношения> -> < | <= | = | <> | > | >= 
<операция типа сложения> -> + | - 
<операция типа умножения> -> * | / 
\end{alltt}
\end{framed}

\newpage